\documentclass[12pt]{article}
\usepackage[utf8]{inputenc}

\usepackage{amsmath, amsfonts, amsthm}
\usepackage{subfiles}
\usepackage{todonotes}
\usepackage{comment} 
\usepackage{epigraph}
\usepackage{hyperref}
\hypersetup{
	colorlinks=true,
	linkcolor=blue,
	filecolor=magenta,      
	urlcolor=cyan,
}

\urlstyle{same}

\theoremstyle{definition}
\newtheorem{definition}{Definition}[section]
\numberwithin{definition}{subsection}

\newtheorem{theorem}{Theorem}
\numberwithin{theorem}{subsection}

\newtheorem{corollary}{Corollary}
\numberwithin{corollary}{subsection}

\newtheorem{result}{Result}

\newtheorem{example}{Example}
\numberwithin{example}{subsection}

\theoremstyle{remark}
\newtheorem*{remark}{Remark}

\theoremstyle{point}
\newtheorem*{point}{\textbf{Important Point}}

\newcommand{\R}{\mathbb{R}}
\newcommand{\todog}{\todo[inline, color=green!40]}
\newcommand{\newchapter}{\newpage\subfile}

\title{A User's Guide to Calculus in $\mathbb{R}^n$}
\author{Shimal Harichurn}

\begin{document}
	\maketitle
	\begin{abstract}
		\begin{center}
			This is a set of notes that I'm compiling on my way to understanding and learning Calculus in $\mathbb{R}^n$. Think of this as a user's guide to Calculus in $\mathbb{R}^n$.
		\end{center}
	\end{abstract}
	
	\section{Introduction}
		
	Calculus is hard. I'll be honest I understand more about General Topology (even Algebraic Topology) than I do basic Calculus at this point in time. Heck I think I even understand (truly) basic Category Theory better than basic Calculus at this point in time. \\
		
	The reason for this, is I think the folllowing; Topology and Algebra are fields (no pun intended) of mathematics that are very structural in nature to me. You take a set, put a topology on it and voila you have a topological space. \\
			
	Calculus is different, there's no real inherent structure (to me at this point in time).
	
	\newpage
	
	\tableofcontents
	
	\newchapter{chapters/chapter1}	
	\newchapter{chapters/chapter2}
	\newchapter{chapters/chapter3}	
	\newchapter{chapters/chapter4}	
	\newchapter{chapters/chapter5}	
	
	\bibliography{bibliography}{}
	\bibliographystyle{plain}
	

    
\end{document}